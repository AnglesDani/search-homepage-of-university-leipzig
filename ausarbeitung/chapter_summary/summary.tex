Dieser Praktikumsbericht behandelt die Erstellung einer Suchmaschine im Rahmen der Information-Retrieval-Vorlesung.
Die Domäne wurde auf die Universität Leipzig eingeschränkt.\\
Während der Bearbeitung der Praktikumsaufgabe konnten viele für die Entwicklung einer Suchmaschine
typischen Komponenten angefertigt und verwendet werden.
Dieser Bericht beleuchtet entsprechend einer für Suchmaschinen üblichen Architektur~\cite{croft.chap2} das
Vorgehen dafür.\\
Dementsprechend beschreibt Kapitel~\ref{chap:data_aqcuisition} den Indexierungsprozess.
Dieser ermöglichte die Erstellung eines Index über 390 000 Dokumente.\\
Kapitel~\ref{chap:query_processing} beschreibt den Anfrageprozess.
Dabei werden die Bestandteile zur Nutzerinteraktion, zum Ranking und für das Logging besprochen.\\
Abgeschlossen wird die Ausarbeitung mit einer Auswertung
der Effektivität der entstandenen Suchmaschine basierend auf einem Laborexperiment (Kapitel~\ref{chap:log_analysis}).\\
Auf Grundlage dieser Auswertung werden potentielle Verbesserungen identifiziert, welche im Anschluss an die Arbeit umgesetzt
werden können.
